% Настройки форматирования документа
% ГОСТ-совместимые настройки

% Параметры страницы
\usepackage[left=3cm,right=1.5cm,top=2cm,bottom=2cm]{geometry} % Поля

% Настройки интервалов и отступов
\linespread{1.5}               % Междустрочный интервал
\setlength{\parindent}{1.0cm}  % Отступ для абзаца (красная строка) ИСПРАВЛЕНО с 1.25 на 1.0 см
\emergencystretch=25pt         % Перенос текста при переполнении

% Настройки шрифтов
% Вариант 1: Стандартные шрифты (работает со всеми компиляторами)
% Раскомментируйте следующий блок для использования Times New Roman
% (Работает ТОЛЬКО с XeLaTeX или LuaLaTeX!)
% -----------------------------------------------------------------
\usepackage{fontspec}
\defaultfontfeatures{Ligatures=TeX}
\setmainfont{Times New Roman}[
  BoldFont       = {Times New Roman Bold},
  ItalicFont     = {Times New Roman Italic},
  BoldItalicFont = {Times New Roman Bold Italic},
  Ligatures      = TeX
]
\setsansfont{Times New Roman}
\setmonofont{Courier New}[Scale=0.9]
% -----------------------------------------------------------------

% Настройки URL-ссылок и гиперссылок
\urlstyle{same}   % Шрифт для URL-ссылок такой же как у основного текста

% Цвет гиперссылок и цитирования
\hypersetup{ 
    colorlinks=true,     % Цветные ссылки вместо рамок
    linkcolor=black,     % Цвет внутренних ссылок
    filecolor=blue,      % Цвет ссылок на файлы
    citecolor=black,     % Цвет ссылок на источники литературы
    urlcolor=blue,       % Цвет URL-ссылок
    pdftitle={},         % Заголовок документа в метаданных PDF
    pdfauthor={},        % Автор документа в метаданных PDF
    pdfsubject={},       % Тема документа в метаданных PDF
    pdfcreator={LaTeX},  % Создатель документа в метаданных PDF
}

% ДОБАВЛЕНО: Настройки заголовков по ГОСТ
\usepackage{titlesec}

% Форматирование заголовков разделов (section)
\titleformat{\section}
  {\normalfont\bfseries\large} % формат: жирный
  {\thesection} % метка
  {1em} % расстояние между меткой и заголовком
  {} % код перед заголовком
  [] % код после заголовка

% Форматирование подзаголовков (subsection)
\titleformat{\subsection}
  {\normalfont\bfseries\itshape\normalsize} % жирный курсив
  {\thesubsection}
  {1em}
  {}
  []

% Отступы до и после заголовков
\titlespacing*{\section}{0pt}{12pt}{12pt}
\titlespacing*{\subsection}{0pt}{12pt}{12pt}

% ДОБАВЛЕНО: Настройки нумерации страниц по ГОСТ
\usepackage{fancyhdr}
\pagestyle{fancy}
\fancyhf{} % очистить все колонтитулы
\fancyfoot[R]{\thepage} % номер страницы в правом нижнем углу
\renewcommand{\headrulewidth}{0pt} % убрать линию в верхнем колонтитуле