% Первый раздел - обзор литературы
\section{ОБЗОР ЛИТЕРАТУРЫ}

\subsection{История изучения макроэргических соединений}
Понятие макроэргических соединений было впервые введено в научный оборот в работах Липмана\cite{example1}. Центральное место среди этих соединений занимает аденозинтрифосфат (АТФ).

\subsection{Основные типы макроэргических соединений}
В клетке существует несколько основных типов макроэргических соединений:
\begin{itemize}
    \item Нуклеозидтрифосфаты (АТФ, ГТФ, ЦТФ, УТФ)
    \item Фосфоенолпируват
    \item Креатинфосфат
    \item Ацетил-КоА
\end{itemize}

\subsection{Механизмы синтеза АТФ}
Синтез АТФ может происходить несколькими путями:
\begin{enumerate}
    \item Окислительное фосфорилирование в митохондриях
    \item Субстратное фосфорилирование в гликолизе и цикле Кребса
    \item Фотофосфорилирование в хлоропластах
\end{enumerate}

% Здесь можно продолжить раздел...