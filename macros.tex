% Макросы и переопределения для удобства использования

% Стиль вложенного нумерованного списка 
% Уровень 1: просто арабские цифры (1, 2, 3...)
\renewcommand{\theenumi}{\arabic{enumi}}
\renewcommand{\labelenumi}{\theenumi.}

% Уровень 2: использует значение первого уровня (1.1, 1.2, 2.1...)
\renewcommand{\theenumii}{\theenumi.\arabic{enumii}}
\renewcommand{\labelenumii}{\theenumii.}

% Уровень 3: использует значение второго уровня (1.1.1, 1.1.2, 1.2.1...)
\renewcommand{\theenumiii}{\theenumii.\arabic{enumiii}}
\renewcommand{\labelenumiii}{\theenumiii.}

% Уровень 4 (при необходимости)
\renewcommand{\theenumiv}{\theenumiii.\arabic{enumiv}}
\renewcommand{\labelenumiv}{\theenumiv.}

% Полезные команды для химии и биохимии
\newcommand{\nadh}{\(\text{НАД}\! \cdot\! \text{H}\ \)}
\newcommand{\nadp}{\(\text{НАД}^+\ \)}
\newcommand{\nadph}{\(\text{НАДФ}\! \cdot\! \text{H}\ \)}

% Пример для рисунков
\newcommand{\insertimage}[3][1.0]{
    \begin{figure}[ht]
        \centering
        \includegraphics[width=#1\textwidth]{#2}
        \caption{#3}
        \label{fig:#2}
    \end{figure}
}

% Пример для таблиц
\newcommand{\inserttable}[2]{
    \begin{table}[ht]
        \centering
        \caption{#1}
        \label{tab:#2}
        \begin{tabular}{#2}
            % Содержимое таблицы
        \end{tabular}
    \end{table}
}

% Команда для важных заметок
\newcommand{\note}[1]{
    \begin{center}
        \fbox{\begin{minipage}{0.9\textwidth}
            \textbf{Примечание:} #1
        \end{minipage}}
    \end{center}
}