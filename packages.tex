% Поддержка русского и английского языков, кодировок и шрифтов
\usepackage[T2A]{fontenc}
\usepackage[utf8]{inputenc}
\usepackage[english, russian]{babel}

% Математические символы и уравнения
\usepackage{amsmath}
\usepackage{amssymb}
\usepackage{amsthm}
\usepackage{gensymb}
\usepackage{mathtools}

% Графика и рисунки
\usepackage{graphicx}
\usepackage{svg}
\usepackage{pst-all}
\usepackage{pstricks-add}
\usepackage{caption}
\usepackage{xcolor}

% Гиперссылки и навигация
\usepackage{hyperref}

% Форматирование текста
\usepackage{indentfirst}
\usepackage{setspace}

% Управление нумерацией
\usepackage{chngcntr}
\numberwithin{equation}{section}

% Листинг кода
\usepackage{listings}

% Создание несколько колонок текста
\usepackage{multicol}

% Цитирования (снимите комментарий с необходимого)
% \usepackage[backend=biber, style=gost-numeric, language=auto, autolang=other]{biblatex} % Сортировка по умолчанию (по алфавиту)
\usepackage[backend=biber, style=gost-numeric, sorting=none, language=auto, autolang=other]{biblatex} % сортировка по мере появления в тексте
\addbibresource{bibsource.bib} % файл с ссылками на статьи из google scholar 

% Пакет import для продвинутой работы с подключением файлов
\usepackage{import}

% Пакет для работы с файловой системой
\usepackage{xstring}

% Команда для автоматического подключения всех файлов из директории content
\newcommand{\importcontentfiles}{%
  % Чтение списка файлов из внешнего файла filelist.tex
  % Этот файл содержит список входных файлов в нужном порядке
% Изменяйте его для контроля порядка разделов в итоговом документе

% Введение
% Введение
\section{ВВЕДЕНИЕ}

В данной работе рассматривается важная научная проблема, связанная с регуляцией производства и потребления макроэргических молекул в клетке.

Актуальность темы определяется необходимостью более глубокого понимания механизмов энергетического обмена на клеточном уровне.

Цели данной работы:
\begin{enumerate}
    \item Анализ существующих механизмов регуляции синтеза АТФ
    \item Исследование взаимосвязи между энергетическим статусом клетки и активностью метаболических путей
    \item Оценка возможностей терапевтического воздействия на процессы энергетического обмена
\end{enumerate}

% Здесь можно продолжить введение...

% Основные разделы - используйте существующие файлы или создайте новые
% Первый раздел - обзор литературы
\section{ОБЗОР ЛИТЕРАТУРЫ}

\subsection{История изучения макроэргических соединений}
Понятие макроэргических соединений было впервые введено в научный оборот в работах Липмана\cite{example1}. Центральное место среди этих соединений занимает аденозинтрифосфат (АТФ).

\subsection{Основные типы макроэргических соединений}
В клетке существует несколько основных типов макроэргических соединений:
\begin{itemize}
    \item Нуклеозидтрифосфаты (АТФ, ГТФ, ЦТФ, УТФ)
    \item Фосфоенолпируват
    \item Креатинфосфат
    \item Ацетил-КоА
\end{itemize}

\subsection{Механизмы синтеза АТФ}
Синтез АТФ может происходить несколькими путями:
\begin{enumerate}
    \item Окислительное фосфорилирование в митохондриях
    \item Субстратное фосфорилирование в гликолизе и цикле Кребса
    \item Фотофосфорилирование в хлоропластах
\end{enumerate}

% Здесь можно продолжить раздел...

% Для добавления новых разделов создайте файлы в папке content/
% и раскомментируйте/измените соответствующие строки:

% Пример использования команды cite с номером страницы
\section{ПРИМЕРЫ БИБЛИОГРАФИЧЕСКИХ ССЫЛОК}

\subsection{Правильное оформление ссылок по ГОСТ}

% Ссылка на источник без указания страницы
По данным недавних исследований \cite{example1}, уровень АТФ в клетке может меняться в зависимости от внешних условий.

% Ссылка на источник с указанием страницы
Согласно работе Сидорова \cite[с.~42]{example2}, основные механизмы регуляции энергетического обмена включают аллостерическую регуляцию ферментов.

% Ссылка на несколько источников
В ряде работ \cite{example1,example3} показано влияние кислородного режима на синтез макроэргических соединений.

% Ссылка на источник с указанием нескольких страниц
Более подробно этот вопрос рассмотрен в монографии \cite[с.~42--47]{example2}.  % раскомментируйте и создайте файл
% \input{content/03-third-section}   % раскомментируйте и создайте файл
% \input{content/04-fourth-section}  % раскомментируйте и создайте файл

% Заключение (создайте файл перед использованием)
% \input{content/99-conclusion}
}